\documentclass{article}
\usepackage{graphicx}

\usepackage[utf8]{inputenc}
\usepackage{amsfonts}
\usepackage{bbold}
\usepackage{amsmath}
\usepackage[english]{babel}
\usepackage[T1]{fontenc}
\usepackage{amsthm}
\usepackage{hyperref}
\usepackage{graphicx}
\usepackage{multirow}
\usepackage{subfig}
\hypersetup{
    colorlinks=true,
    linkcolor=cyan,
    citecolor=blue,    
    urlcolor=red,
    pdfpagemode=FullScreen,
    }
\usepackage{graphicx}
\usepackage{array}

\newtheorem{theorem}{Theorem}
\newtheorem{lemma}{Lemma}
\newtheorem{prop}{Proposition}
\newtheorem{corro}{Corrolary}

\title{Memorization in Attention-only Transformers}
\author{Léo Dana}
\date{\today}

\begin{document}

\maketitle


\begin{abstract}
    Summary
\end{abstract}

\section{Introduction}

Several papers have shown that memorization, especially of factual information, happens in the MLPs of a Transformer
[ROME, MEMIT, Fact Finding]. 
\bigbreak

In this picture, attention layers are considered to solely move information between each token's stream [Look Before You Leap].
\bigbreak

But is this all that attention do regarding memorization ? Although it has not been shown that attention layers are memorizing
empirically, one can wonder if they have the capacity to do so. And if so, what is the algorithm implemented by the attention ?
\bigbreak

To answer these questions, we focus on an Attention-only Transformer (AoT) with only 1 layer and no MLP, so that only the attention memorizes.
\bigbreak

We call an AoT with dimension $d$ "fully expressive" if it can represent any linear map of rank $d$. When a Transformer is fully 
expressive, this means we only need to understand the best linear mappings. 
\bigbreak

Our main results are:
\begin{itemize}
    \item A proof, that, given sufficiently many attention modules, the AoT can be fully expressive.
    \item Bounds on the KL divergence of the best linear map with $\pi$ in two cases: when the distribution to learn has rank almost $d$,
    and when the distribution is a look-up table. For the former, memorizing means exaclty learning the $d$ most important singular directions,
    for the latter it means memorizing every example but with some noise.
\end{itemize}


\section{Related work}

Associative memory: in there paper, they present a model of memory and show that the model can memorize at most $d$ features. 
Yet, there approach of memory seems to be valid only in the regime where $d$ is close to $N$. Thus we want to know what is the more general
solution when $d$ is much lower. Works have shown that models can have an exponential memory using superposition. 
To study these phenomenons, we generalize the papers' setting. This makes it hard to compare results.
For example, on there task, our model doesn't exhibit a maximum storage capacity, which makes the comparison meaningless. 
\bigbreak

Memorization in the MLP -> several papers have shown that memorization in real transformers happend in MLPs (ROME / MEMIT / Factual recall), 
but what about memorization in Transformers ? 
\bigbreak

Attention Only Transformers -> the mathematical equivalence between Transformers and Attention only Transformers has been shown in X, Y.
Yet there construction are very costly because one simulate an MLP using an Attention instead of using fully the attentional representation
power. We will prove that MLP and Attention layers are equaly good when learning distribution.
\bigbreak

Low-Rank Bottleneck in Multi-head Attention Models -> They prove that with $d<S$ the Transformer cannot produce arbitrary attention pattern
and thus cannot become fully expressive. We provide a proof that if the total cumulated head dimension is large enough, then this 
restriction disappear.
\bigbreak

On the Expressive Power of Self-Attention Matrices -> They show that it is sufficient to have $k^2\log(S)$ head dimension to approximate
attention pattern which are $k$ sparce. Theorem X in appendix also approximate sparce attention pattern but uses $S$ dimension to do so.
Combining both ideas could provide a better approximation theorem !
\bigbreak

\section{Attention-only Transformers}

Let $\pi$ a distribution on sequences of $N$ possible tokens, of lenght $S$. We generate each new token conditional on the 
last $k$ \[t_{i}\sim \pi(\cdot\,|t_{i-1:i-k})\] We call $\pi(t_{i-1:i-k})$ the prior distribution and 
$\pi(\cdot\,|t_{i-1:i-k})$ the conditional distribution. We take the conditionnal distribution such that it doesn't have any zero 
probabilities. Most results will generalize when the conditional contains some zeros, we describe the changes in appendix ???.
\bigbreak

We are interested in the problem of learning a function to approximate the distribution $\pi$ for token of index $>k$. Our goal is to 
minimize the $KL$-divergence for sequences of lenght $S$, defined as 
\[d_{KL}^S(\pi, f) = \mathbb{E}_{t\sim\pi}\left[\frac{1}{S-k}\sum_{i=k+1}^S\log\left(\frac{\pi(\cdot|t_{i-1:i-k})}{f(t_{1:i-1})}\right)\right]\]



\subsection{Attention-only Transformer's Formalism}

We consider a variant of the transformer architecture $T$ with only 1 layer, 1 head per attention mechanism, but $P$ attention layers 
computed in parallel containing one heads each. Although all results will be stated for one head per attention module, 
they hold for any number of heads. The embedding dimension is $d$.\\
Let $t_{1:S} = \{t_i\}_{i=1:S}$ a sequence of $S$ tokens. The transformer's computation writes as follows,
\[T(t_{1:S})_s = W_U\left(e(t_s)+\sum_{p=1}^PW_{OV}^{p}A^p(t_{1:S})\right)_s\] where each Attention block can be written 
\[A^{p}(t_{1:S})_s = \sum_{j=1}^sa_s^{p}(t_{1:S})_j(e(t_j)+pos_j)\] and $a$ is the softmaxed attention score
\[a_s^{p}(T_{i}(t_{1:S}))_j = \text{softmax}\left(\frac{1}{\sqrt{d}}(e(t_s)+pos_s)^TW_{QK}^{p}(e(t_i)+pos_i), \, i=1:s\right)_j\] 
$pos_s$ are the additive positional embeddings, $e$ is the token embedding matrix and $W_U$ the unembedding matrix. We have factorized matrices 
$W_Q^p$ and $W_K^p$ as well as $W_V^p$ and $W_O^p$, following the intuition from [Mathematical framework for Transformers].
\bigbreak

By concatenating the matrices $W_{OV}^p$ and $A^p$, we can factorize the attention asa sum of linear maps,
\[T(t_{1:S})_s = W_U(e(t_{s}) + W_{OV}A(t_{1:S})_s)\]
with $W_{OV}\in \mathbb{R}^{d, dP} $ and $A(t_{1:S})\in \mathbb{R}^{dP}$. 
\bigbreak

We also introduce $f_W(t_{1:S}) = \text{softmax}(W\text{onehot}(t_{1:S}))$, the low rank linear map on token sequences.
We can rewrite $T = f_{W_UW_E}$, were $W_E$ depends on $e$, $pos$, $W_{OV}^p$ and $W_{QK}^p$. We say that an AoT is 
\textit{fully expressive} when its architecture can represent any low rank linear map.

\subsection{Universal approximation}

Since any AoT can be seen as a low rank linear map from the token sequences to logit, and so are Transformers in general, 
the class of function is at least as expressive as the class of linear map.
\bigbreak
\begin{prop}
    Let $T$ be any transformer with embedding dimension $d$, and let $W^*$ of rank $d$ minimizing the KL-divergence with $\pi$.
    Then we have that for any $S\geq k+1$, \[d_{KL}^S(\pi, T) \geq d_{KL}^{k+1}(\pi, f_{W^*}) \]
    \bigbreak
\end{prop}

In fact, for the class of AoT, they are as expressive as the class of low rank linear maps. 
Moreover, we can control the approximation error when the AoT doesn't have enough heads to be fully expressive.
\bigbreak
\begin{theorem}
    Let $\epsilon \geq 0$, $N_{\varepsilon}$ the smallest number of questions whose cumulative probability is greater than 
    $1-\varepsilon$, and $f_{W^*}$ the optimal linear mapping of rank $d$. 
    There exist an AoT $T^*$ with embedding dimension $d$ and $\lceil\frac{N_{\epsilon}}{d}\rceil$ total parallel attention modules 
    such that its divergence with $\pi$ is bounded
    \[\left|d_{KL}^{k+1}(\pi, f_{W^*}) - d_{KL}^{k+1}(\pi, T^*)\right| 
    \leq 
    \epsilon\sigma_1(f_{W^*})C(d,N,SN_{\epsilon})\]
    \bigbreak
\end{theorem}
\noindent Here, taking $\varepsilon=0$ gives a fully expressive AoT Transformer. In the bound, the constant $\sigma_1(f_{W^*})$
means the biggest prediction made by $f_{W^*}$, since in the worst case, by not controling the predictions of the low probability 
sequences, the AoT might predict a logit of very high norm.\\
This theorem holds if we try to predict several tokens one after the other, meaning that we bound $d_{KL}^{S}(\pi, T^*)$ at the cost 
of having a bigger residual stream.
\bigbreak

\bigbreak

[TODO] Learning with MLPs uses the same number of parameters, and has a lower bound. This means that attention is actually as expressive as MLPs.
\bigbreak

\section{Optimal Linear Mappings}

Let $f:\mathbb{R}^N \rightarrow \mathbb{R}^M$ a linear function of rank $d$, $S$  the softmax function, and $\pi\in\Delta^{N,M}$ a bigram 
distribution. We want to find the optimal mapping $f$ that minimizes the $KL$-divergence 
$d_{KL}(\pi, S\circ f) = d_{KL}(\pi, f) = \mathbb{E}_{t_i,t_o}\left[\log\left(\frac{\pi(t_o|t_i)}{(S\circ f)(t_i)_{t_o}}\right)\right]$. We call it the 
\textit{Bigram Problem}: given an inpout token in $[N]$ the task is to predict the correct probability distribution on output tokens in $[M]$. 
\bigbreak

\noindent \textbf{Notations}: $t_i$ always represent the index of the input token and $t_o$ of the output token. We denote $\pi_{t_i}$ 
the conditional distribution $(\pi(t_o|t_i))_{t_o}$, $\Pi_{t_i,t_o}=\pi(t_o|t_i)$ the matrix whose columns are the conditional probabilities, 
and similarly $L_{t_i,t_o}=\log(\pi(t_o|t_i))$. We let $I_{\pi_{t_i}}$ be the diagonal matrix with value $\pi(t_o|t_i)$ and let $H(\pi)$ the 
entropy of $\pi$.
\bigbreak

The solutions to this problem depend on the rank of $L - \mathbb{E}[L]$, which has coefficient $L_{t_i,t_o} - \mathbb{E}_{t_o}[L_{t_i,t_o}]$. 
In particular, if $d \geq rank(L-\mathbb{E}[L])$, then we can take 
$f = L-\mathbb{E}[L]$ which give 0 divergence. Otherwise, the problem is harder to solve because of the non-linearity of the softmax.
To go one step further, we rewrite the divergence as \[d_{KL}(\pi, f) = \mathbb{E}_{t_i}\left[\log\left(\mathbb{E}_{t_o}\left[e^{\Bar{f}(t_i)_{t_o}}\right]\right)\right]\] with 
$\Bar{f}(t_i)_{t_o} = f(t_i)_{t_o}-L_{t_i,t_o} - \mathbb{E}_{t_o}[(t_i)_{t_o}-L_{t_i,t_o}]$. Note that this expression shows we can choose $w_E(t_i)$ independantly for
each $t_i$ by minimizing the sum of exponential. This will be the basis of our analysis. To find $W_U$, however, requires to find the matrice that 
take into account all the prior and the posterior probability $\pi$. In very specific case, the optimal $W_U$ can be guessed.
\bigbreak

The following sections propose progress on two settings: when $L-\mathbb{E}[L]$ has almost rank $d$, meaning that $L$'s $d+1$ largest singular 
value is small, and when $\pi$ has low entropy, meaning it is almost a one-to-one mapping.
\bigbreak

\subsection{Almost rank \textit{d}}

Let $\sigma_j$ the $j$-th largest singular value of $L - \mathbb{E}[L]$. 
We treat here the case where $\sigma_{d+1}$ is small. Intuitively, this means that solving the least-square problem will give an
a good solution to the problem. We use this fact to obtain a bound on the divergence in the Lemma below.
\bigbreak

\begin{lemma}
    Let \[\Bar{f}(t_i)_{t_o} = f(t_i)_{t_o}-L_{t_i,t_o} - \mathbb{E}_{t_o}[(t_i)_{t_o}-L_{t_i,t_o}]\] 
    we have the following bound on the $KL$-divergence
    \[\left|d_{KL}(\pi, f) - \mathbb{E}_{t_i}\left[\log\left(1+\frac{1}{2}\mathbb{V}_{t_o}\left(\Bar{f}(t_i)_{t_o}\right)\right)\right]\right| 
    \leq 
    \mathbb{E}_{t_i}[\exp_3(||\Bar{f}(t_i)||_2)]\]
    with $\exp_3(x) = e^x-1-x-\frac{x^2}{2}$.
\end{lemma}
\bigbreak

Lemma 1 gives a bound valid when the $L^2$ norm of $\Bar{f}(t_i)$ is small. Yet, in the above equation, $\exp_3(\mathbb{E}_{t_i}[||\Bar{f}(t_i)||_2])$ 
is of order 3 while $\mathbb{V}_{t_o}\left(\Bar{f}(t_i)_{t_o}\right)$ is of order 2. We would like to optimize for the highest order term in order to have 
the most precise bound. We thus propose to choose $f$ by optimizing $\mathbb{V}_{t_o}\left(\Bar{f}(t_i)_{t_o}\right)$ which is also a least-square problem.
\bigbreak

\begin{lemma}
    For each $t_i$, and $f = W_U^TW_E$, the function 
    \[w_E(t_i)\rightarrow \mathbb{V}_{t_o}(\Bar{f}(t_i)_{t_o}) = ||I_{\sqrt{\pi_{t_i}}}(I_d - \mathbb{1}\Pi_{t_i}^T)(W_Uw_E({t_i})-L_{t_i})||_2^2\]
    is minimized by taking $\mathbb{1}\notin Im(W_U)$ and \[w_E({t_i}) = [W_U^T(I_{\pi_{t_i}}-\Pi_{t_i}\Pi_{t_i}^T)W_U]^{-1}W_U^T(I_{\pi_{t_i}}-\Pi_{t_i}\Pi_{t_i}^T)L_{t_i}\]
    We end up with the function 
    \[f({t_i}) = P_{t_i}L_{t_i}, \,\,\,\,\,\, P_{t_i} = W_U[W_U^T(I_{\pi_{t_i}}-\Pi_{t_i}\Pi_{t_i}^T)W_U]^{-1}W_U^T(I_{\pi_{t_i}}-\Pi_{t_i}\Pi_{t_i}^T)\]
    where $P_{t_i}$ is a projection.
\end{lemma}
\bigbreak

Now, using Lemma 2, we can derive a bound on the $KL$-divergence which is in some cases better. 
\bigbreak

\begin{theorem}
    For $f_{ls}$ being the least-square solution of the problem $||f({t_i})-L_{t_i}||_2$, and $C = \left(1+\frac{N||\pi_{t_i}||_{+\infty}}{\sqrt{2}}\right)$, we have the bound 
    \[d_{KL}(\pi, f_{ls}) \leq \mathbb{E}_{{t_i}}\left[\log\left(1+\frac{||\pi_{t_i}||_{+\infty}}{2}C\sigma_{d+1}^2\right)
    + \exp_3\left(\sqrt{C}\sigma_{d+1}\right)\right]\]
    For $f_{wls}$ being the weighted least-square solution of Lemma 2, we have the bound
    \[d_{KL}(\pi, f_{wls}) \leq \mathbb{E}_{{t_i}}\left[\log\left(1+\frac{||\pi_{t_i}||_{+\infty}}{2}\sigma_{d+1}^2\right)
    + \exp_3\left(\sqrt{\frac{||\pi_{t_i}||_{+\infty}}{||\pi_{t_i}||_{-\infty}}}\sigma_{d+1}\right)\right]\]
    Since the choice of $W_U$ is the same in both cases, we can choose the most optimal $w_E({t_i})$ between the two bounds, so taking the 
    minimum of the bounds in the expectation over ${t_i}$ still holds.
\end{theorem}
\bigbreak

\textbf{Remarks}: The above theorem proposes two bounds, which are in general tradeoffs depending on the shape each distributions $\pi_{t_i}$. 
If we have $||\pi_{t_i}||_{-\infty} \simeq \frac{C}{N}$ the optimal choice if to take $f_{wls}$, and if 
$||\pi_{t_i}||_{+\infty} \simeq \frac{C}{N}$ then the optimal choice is $f_{ls}$.
For langage modeling, and with huge dictionnary sizes, we always end up in the situation where $||\pi_{t_i}||_{-\infty} \simeq 0$, so 
in the absence of a titgher bound not involving $||\pi_{t_i}||_{-\infty}$, the weighted least-square solution should be expected. 
\bigbreak

To understand what the solution to the weigthed least-square do is quite tricky in the general case. Corrolary 1 presents a special
case in which the difference between the two proposed solution is easily understandable. This case is also very nice since one doesn't 
have the term in $||\pi_{t_i}||_{-\infty}$. 
\bigbreak

\begin{corro}
    Let $L-\mathbb{E}[L] = V\Sigma U$ the singular value decomposition, and suppose that for $\xi$ a permutation we have $V=I_{\xi}$ a permutation matrice. 
    Let $S_{d}(\xi) = \{\xi(j), 1\leq j\leq d\}$, $\pi(S|{t_i}) = \sum_{{t_o}\in S}\pi({t_o}|{t_i})$, 
    and $H(\pi_{t_i}, S) = \frac{1}{\pi(S|{t_i})}\sum_{{t_o}\in S}\pi({t_o}|{t_i})\log(\pi({t_o}|{t_i}))$
    Then we have
    \[f_{wls}({t_i})_{t_o} = \begin{cases}\log(\pi({t_o}|{t_i}))\,\,\,\,\,\,\,\,\,\,\,\,\,\,\,\,\,\,\,\,\,\,\,\,\,\,\,\,\,\,\,\,\,\,\,\,\,\,\,\,\,\,\,\,\,\,\,\,\,\,\,\,\,\,\,\,\, \text{ if } {t_o}\in S_{d}(\xi)\\
        H(\pi_{t_i}, S_{d}(\xi)^c)+\log(\pi(S_{d}(\xi)^c|{t_i}))\text{ if } {t_o}\in S_{d}(\xi)^c\\ \end{cases}\]
    and the divergence is exactly
    \[d_{KL}(\pi, f_{wls}) = \mathbb{E}_{t_i}\left[\log\left(\pi(S_{d}(\xi)|{t_i})+\pi(S_{d}(\xi)^c|{t_i})(N-d)e^{H(\pi_{t_i}, S_{d}(\xi)^c)}\right)\right] \]
    while
    \[f_{ls}({t_i})_{t_o} = \begin{cases}\log(\pi({t_o}|{t_i})) \text{ if } {t_o}\in S_{d}(\xi)\\
        H(\pi_{t_i})\,\,\,\,\,\,\,\,\,\,\,\,\,\text{ if } {t_o}\in S_{d}(\xi)^c\\ \end{cases}\]
    and
    \[d_{KL}(\pi, f_{ls}) = \mathbb{E}_{t_i}\left[\log\left(\pi(S_{d}(\xi)|{t_i})+(N-d)e^{H(\pi_{t_i})}\right)\right] \]
\end{corro}
\bigbreak

\textbf{Remarks}: Corrolary 1 states that when the singular values are aligne with the output space, then $f_{ls}$ and $f_{wls}$ implement 
exact memorization of the $d$ most important output coordinates. The only difference in that case is that $f_{wls}$ predicts for the 
other tokens a uniform probability over the local mean $H(\pi_{t_i}, S_{d}(\xi)^c)$ of these tokens, whereas $f_{ls}$ predicts the
global mean $H(\pi_{t_i})$, failing to use adapt to the already learned features.\\
In this special case, the weighted least-square solution is strictly better than the least-square.
\bigbreak

\subsection{Look-up table}

When the distribution has low-entropy, this means that their is one next-token to predict while the other are mostly noise. 
If the entropy of the conditional distributions are all exactly 0, then $\pi({t_o}|{t_i}) = \mathbb{1}_{g({t_i})={t_o}}$ for some 
function $g: [N]\rightarrow [M]$. 
In that setting, there exists a sequence of mappings which converges to a 0 divrgence.
This means that despite having an embedding dimension of $d\geq 2$, we can remember an arbitrarely large mapping.
\bigbreak

When the entropy of the conditional distribution is not 0, but is low, we have $\pi({t_o}|{t_i}) \simeq \mathbb{1}_{g({t_i})={t_o}}$. 
Here we can remember exaclty the most important prediction, and uniformly approximate the rest. The following theorem is a bound based 
on this intuition.
\bigbreak

\begin{theorem}
    Let $g:[N] \rightarrow [M]$. There exists $W_U$ be such that \[||W_UW_U^T-I_d||_{\infty} \leq C = \sqrt{\frac{32\log(M+1)}{d}}\]
    We choose $f({t_i}) = \lambda({t_i})W_U^Tw_U(g({t_i}))$, with $\lambda({t_i})$ the solution to the equation 
    \[-H(\pi_{t_i}) = \log\left(\sum_je^{\lambda({t_i})(w_U(j)-w_U(g({t_i})))^Tw_U(g({t_i}))}\right)\] 
    we obtain the bound 
    \[d_{KL}(\pi, f) \leq \mathbb{E}_{t_i}\left[(1-\pi(g({t_i})|{t_i}))\log\left(\frac{M-1}{e^{-H(\pi_{t_i})}-1}\right)
    \left(\frac{1+2C+Cd_{TV}(\tilde{\pi}_{t_i}, \pi_{\text{unif}})}{1-2C}\right)\right]\]
\end{theorem}
\bigbreak

Here we make use of the Johnson-Linderstrauss theorem to make the vector from $W_U$ interact as slightly as possible.

\section{Conclusion}
TODO


\newpage



\appendix




\section{Proof of Universal approximations}

To prove Theorem 1, we will first show that the Attention matrix, with enough heads can be chosen invertible.

\begin{lemma}[Rank of the attention matrix]
    Let $L$ the size of the sequence, $d$ the embedding dimension, and $P$ the number of parallel attention heads. Let $A\in\mathbb{R}^{N^L, Pd}$ as defined in section 3.1.
    For $Pd \geq N^k$, there exists matrices $W_{QK}^p$ and embeddings $e(t_j)$ and $pos_j$ such that $A$ is invertible.
\end{lemma}


\begin{proof}[Proof : Lemma 4]
    Let us choose $e$ and $pos$ with the wollowing properties
    \begin{enumerate}
        \item They have strictly positive cefficients,
        \item For any non-zero polynomial with coefficient in $\mathbb{Z}$ of degree $d$ on $(N+L)d$ variables, then the polynomial is non zero
        on the $(N+L)d$ coefficients of $e(t_j)$ and $pos_j$,
    \end{enumerate}
    Such embeddings exists using transcendental numbers.
    We note $x(t, j) = e(t_j)+pos_j$, and define 
    \[f:W \rightarrow \left(\frac{\sum_{i}^Le^{x(t,L)^TWx(t,i)}x(t,i)}{\sum_{i}^Le^{x(t,L)^TWx(t,i)}}\right)_{t\in\mathcal{T}_L}\]
    To prove that $A$ is invertible is equivalent to showing that its rows span $\mathbb{R}^{N^L}$, and thus to showing that the image 
    of $f$ is not contained in any hyperplan.
    \bigbreak

    Let us prove this by absurd and take $(v_t)_{t\in\mathcal{T}_L}$ such that $\sum_{t\in\mathcal{T}_L}v_t^Tf_{x(t,j)} = 0$ 
    as functional equality. Let $i_{\max}(t) = \arg\max_i(x(t,i)_1)$ the indice of the greatest $x$ on the first dimension.
    We can rewrite the equality as 
    \[\sum_{t\in\mathcal{T}_L}\frac{\sum_{i}^Le^{x(t,L)^TW(x(t,i)-x(t,i_{\max}(t)))}v_t^Tx(t,i)}{\sum_{i}^Le^{x(t,L)^TW(x(t,i)-x(t,i_{\max}(t)))}} = 0\]
    Let us only consider matrices $W$ such that $W_{i,j} = 0$ if $i\neq 1\neq j$. In this case, we have 
    $x(t,L)^TW(x(t,i)-x(t,i_{\max}(t))) = x(t,L)_1^TW_{1,1}(x(t,i)_1-x(t,i_{\max}(t))_1)$ so by definition of $i_{\max}(t)$, this is negative 
    when $W_{1, 1} = w$ is positive. Thus taking $W_{1, 1} \rightarrow +\infty$ makes every exponential go to 0 or stay at 1. Thus,
    for a large enough $w$ we can use the Taylor approximation of $x\rightarrow \frac{1}{1+x}$ around 0. and we have 
    \[\frac{1}{1+\sum_{i\neq i_{\max}(t)}^Le^{x(t,L)^TW(x(t,i)-x(t,i_{\max}(t)))}} = 1+\sum_{j=1}^{+\infty}\left(\sum_{i\neq i_{\max}(t)}^Le^{x(t,L)^TW(x(t,i)-x(t,i_{\max}))}\right)^j\]
    Finally, we let $S(t) = \{i_1, ..., i_i \neq i_{\max}(t), \text{unordered}\}$ and call $N(j_1, ..., j_i)$ the number of 
    possible permutations of this sequence. Now developping the product of sum into a sum of product gives us the following equality
    \[\sum_{t\in\mathcal{T}_L}\sum_{j=0}^{+\infty}\sum_{i_1, ..., i_j \in S(t), i_0}(-1)^jN(j_1, ..., j_i)v_t^Tx(t, i_0)e^{x(t,L)^TW\sum_{k=0}^{j}x(t, i_k) - x(t, i_{max}(t))} = 0\]
    Since the family of exponential function is free, we can identify the coefficient in front of each different exponent as 0.
    \bigbreak

    To end the proof, we show that for each sequence $t$, there exists $d$ coefficients in the exponential that are unique to this 
    token sequence. Consider $n\in [2, d+1]$, and the exponent generated by the $n^i$ times the indice $i\neq i_{\max}(t)$. This
    gives the exponent $x(t,L)_1w\sum_{i\neq i_{\max}(t)}n^i(x(t,i)_1 - x(t,i_{\max}(t))_1)$. Now let $t'$ another token sequence and 
    $c(t)_i$ other coefficient for the number of time that the indice $i$ was chosen, such that
    \[x(t,L)_1\sum_{i\neq i_{\max}(t)}n^i(x(t,i)_1 - x(t,i_{\max}(t))_1) = x(t',L)_1\sum_{i\neq i_{\max}(t')}c(t')_i(x(t',i)_1 - x(t',i_{\max}(t'))_1)\]
    We can use the property (2) of the embedding to identify the coefficients on each side.
    First let us start with the positional embeddings. We only have each time the coefficients $pos_L^Tpos_i$, so we can identify each. 
    This means that $i_{\max}(t) = i_{\max}(t')$ and that $c(t')_i = n^i$.
    Finaly, to indentify the word embedding, we can start by looking at 
    $e(t_L)_1*\left(\sum_{i\neq i_{\max}(t)}^Ln^i(pos_{i, 1} - pos_{i_{\max}(t), 1})\right)$
    and compare it to its counterpart, this gives us $t_L = t'_L$.
    Then we remove the extra information we already deduced and have
    \[\sum_{i\neq i_{\max}(t)}^Ln^i(e(t_i)_1 - e(t_{i_{\max}(t)})_1) = \sum_{i\neq i_{\max}(t)}^Ln^i(e(t'_j)_1 - e(t'_{i_{\max}(t)})_1)\]
    We can take any word embedding and look at its integer coefficient.
    If that number if negative, it means that this word embedding is also the maximum of the sequence, so it should be as well for
    the sequence of the right. 
    If positive, it is a sum of power of $n$, so it can be written in base $n$ uniquely. This unicity mean 
    that the sequence on the right should produce the same, and thus we have $t_j = t'_j$. 
    If each side is null, this means that every token is the same, but since we already deduced 
    $t_L$ we have all the information we need.
    \bigbreak

    This shows that $t = t'$, that the number of indices is $j = \sum_{i\neq i_{\max}(t)}^Ln^i$ of $j = \sum_{i\neq i_{\max}(t)}^Ln^i+1$
    if $i_0 = i_{\max}(t)$. So we get the coefficient in front of the exponential
    \[v_t^T\sum_{i=1}^{N}x(t,i)N_i = 0\] where $N_i$ is the number of permutation of the sequence $(i_1, ..., i_j)$ with 
    $n^k - \mathbb{1}_{i=k\neq i_{\max}(t)}$ times the indice $k$.
    The matrice of this system has non-zero determinant by property (2), so the system solves for $v_t = 0$.
\end{proof}

\begin{proof}[Proof : Theorem 1]
    Let us first prove the theorem when $\epsilon = 0$. Let $N_0$ be the number of sentences with non-zero prior probability. We take $P = \left\lceil \frac{N_0}{d}\right\rceil$.
    We have $T(t_{1:k}) = W_Ue(t_k) + W_UW_{OV}A(t_{1:k})$, we choose $e$, $pos$ and $W_{QK}^p$ such that $A$ is invertible like in Lemma 4. Now denote $W_U^*$ and $W_E^*$ the 
    matrices that minimize $d_{KL}(\pi,f_{W_U^*W_E^*})$. We take $W_U = W_U^*$, and $W_{OV} = (W_E^* - e\otimes 1 \otimes ... \otimes 1)A^{-1}$, 
    with $(e\otimes 1 \otimes ... \otimes 1)(t_{1:k}) = e(t_k)$, which gives $T^* = W_U^*W_E^*$ the fully expressive Transformer.
    \bigbreak

    TODO
\end{proof}
\bigbreak

\subsection{Auto-regressive AoT}

The setting of the paper focuses on  one single next token prediction. However, one has Theorem 1 for several prediction in a row.
The scaling in $d$ needed is the sequence size, which is worst than [Expressivity of the Attention] to produce the a matrice that 
has the same focus. By using their results, one might trade the scaling $d \sim k^2\log(S)$ for another error term.
\bigbreak

\begin{corro}
    There exist a transformers $T^*_{\lambda}$ with embedding dimension $d\geq S+1$, $\left\lceil\frac{N^k}{d}\right\rceil$ parallel attention module 
    and 1 layer, such that \[d_{KL}^{S}(\pi, T^*_{\lambda}) + o(e^{-\lambda}) = \underset{\text{r}(W)=d}{\min}d_{KL}^{k+1}(\pi, f_W)\]
\end{corro}
\bigbreak

\begin{proof}[Proof : Corrolary 2]
    We keep the same notations as in the proof of Theorem 2, but we let $t_{j:s} = \{t_i\}_{i=j:s}$. We have that for all 
    $k+1\leq s\leq S$\[T_{\lambda}(t_{1:s}) = W_UW_E(t_{s}) + W_UW_O\tilde{a}(t_{1:s})\] and with the same reasoning, it is sufficient to have 
    \[W_E^*(t_{s-k:s}) = W_E(t_{s}) + W_O\tilde{a}(t_{1:s})\] to have optimality of the prediction. Thus, if we can show that 
    \[\tilde{a}(t_{1:s}) = \tilde{a}(t_{s-k:s}) + o(e^{-\lambda})\] this will conclude the proof, as the rest is the same as Theorem 2.
    \bigbreak

    To prove this, we will focus on the attention pattern, and make sure the model only look at the $k$ last tokens only. Let $(e_i)_{i=1:d}$ 
    an orthonormal family of the embedding space, and we define $\mathcal{H}_1 = Vect(e_i, i=1:S-k)$, and $\mathcal{H}_2 = Vect(e_i, i=S-k+1:d)$.
    Let $pos_s = e_s + \tilde{pos}_s$.
    We know that the matrices $W_{QK}^{p,h}$ are of dimension $d'=\frac{d}{H}$. We can thus fix 
    \[W_{QK}^{p,h} = \lambda\sum_{s=k+1}^Se_s\left(\sum_{i=s-k}^se_{i}\right)^T + \tilde{W}_{QK}^{p,h}\] where $\tilde{W}_{QK}^{p,h}$ is of rank
    $d'-S-k$, and of kernel included in $\mathcal{H}_2$. With this choice, we have that 
    $pos_j^TW_{QK}^{p,h}pos_i = \lambda$ if both $k+1\leq j\leq S$ and $j-k\leq i\leq j$, and $0$ otherwise.
    \bigbreak

    For $k+1\leq s \leq S$, and $1\leq i\leq s$, we have that 
    \[(W_E(t_s)+ pos_s)^TW_{QK}^{p,h}(W_E(t_i)+pos_i) = \lambda\delta_{s-k\leq i\leq s} + (W_E(t_s) + \tilde{pos}_s)^T\tilde{W}_{QK}^{p,h}(W_E(t_i)+ \tilde{pos}_i)\] 
    Intuitively, what this construction does is use $S$ dimension to make the network focus on the right tokens at each step, and uses the rest 
    of the dimenions to produce an invertible matrice $\tilde{a}$. Now, we can see that after the softmax, we get 
    \[a^{p,h}(t_{1:s})_j = a^{p,h}(t_{s-k:s})_j(1+o(e^{-\lambda}))+o(e^{-\lambda})\] 
\end{proof}

\section{Proofs for Optimal Linear Mappings}

\begin{proof}[Proof : Lemma 2]
    Recall that $\Bar{f}(t_i)_{t_o} = f({t_i})_{t_o}-\mathbb{E}_{t_o}[f({t_i})]_{t_o} - (L_{{t_i},{t_o}}-\mathbb{E}_{t_o}[L_{{t_i},{t_o}}])$.
    \begin{equation}
        \begin{split}
            d_{KL}(\pi, f) &= \mathbb{E}_{{t_i}}[H(\pi_{t_i})]-\mathbb{E}_{{t_i},p}\left[\log\left(\frac{e^{f({t_i})_p}}{\sum_{t_o}e^{f({t_i})_{t_o}}}\right)\right]\\
            &=\mathbb{E}_{{t_i}}[H(\pi_{t_i})]-\mathbb{E}_{{t_i},p}[f({t_i})_p-\mathbb{E}_{t_o}[f({t_i})_{t_o}-L_{{t_i},{t_o}}]]\\
            &-\mathbb{E}_{{t_i}}\left[\log\left(\sum_{t_o}e^{f({t_i})_{t_o}-\mathbb{E}_{t_o}[f({t_i})_{t_o}-L_{{t_i},{t_o}}]}\right)\right]\\
            &=\mathbb{E}_{{t_i}}\left[\log\left(\sum_{t_o}\pi({t_o}|{t_i})e^{\Bar{f}(t_i)_{t_o}}\right)\right] \\
            &= \mathbb{E}_{t_i}[\log(1+\mathbb{E}_{t_o}[e^{\Bar{f}(t_i)_{t_o}}-1|{t_i}])]
        \end{split}
    \end{equation}
    For the bound we use the finite increment inequality for $x\rightarrow \log(1+x)$ 
    \[\left|\frac{\log(1+y)-\log(1+x)}{y-x}\right| \leq \frac{1}{1+\min(y,x)} \leq 1\] with $y = \mathbb{E}_{t_o}[e^{\Bar{f}(t_i)_{t_o}}-1]$ and 
    $x = \frac{1}{2}\mathbb{V}_{t_o}\left(\Bar{f}(t_i)_{t_o}\right)$ both positive.
\end{proof}

\bigbreak

\begin{proof}[Proof : Lemma 3] 
    We decompose the variance as
    \begin{equation}
        \begin{split}
            \mathbb{V}_{t_o}(\Bar{f}(t_i)_{t_o}) &= \mathbb{V}_{t_o}(f({t_i})_{t_o}) -2Cov_{t_o}(L_{{t_o}},f({t_i})_{t_o}) + \mathbb{V}_{t_o}(L_{{t_o}})\\
            &= w_E^T({t_i})\mathbb{V}_{t_o}(w_U({t_o}))w_E({t_i}) - 2Cov_{t_o}(L_{{t_o}},w_U^T({t_o}))w_E({t_i}) + \mathbb{V}_{t_o}(L_{{t_o}})\\
        \end{split}
    \end{equation}
    by taking the gradient on $w_E({t_i})$, we get the equation \[\mathbb{V}_{t_o}(w_U({t_o}))w_E({t_i}) = Cov_{t_o}(L_{{t_o}},w_U({t_o}))\] 
    So using that $\mathbb{V}_{t_o}(w_U({t_o})) = W_U(I_{\pi}-\Pi\Pi^T)W_U^T$ is symetric and invertible if $\mathbb{1}\notin Im(W_U^T)$, which is a 
    reasonnable assumption to make: since the softmax function erase the direction $Vect(\mathbb{1})$, $W_U^T$ has no reason to learn this in 
    its span. We find \[w_E({t_i}) = \mathbb{V}_{t_o}(w_U({t_o}))^{-1}Cov_{t_o}(L_{{t_o}},w_U({t_o}))\]
\end{proof}

\bigbreak

\begin{proof}[Proof : Theorem 2]
    \begin{equation*}
        \begin{split}
            d_{KL}(\pi, f) & = \mathbb{E}_{t_i}[H(\pi_{t_i})] - \mathbb{E}_{{t_i},{t_o}}\left[\log\left(\frac{f_{t_o}({t_i})}{\sum_je^{f_j({t_i})}}\right)\right]\\
            &= \mathbb{E}_{t_i}[H(\pi_{t_i})] + \mathbb{E}_{{t_i},{t_o}}\left[\log\left(\sum_je^{f_j({t_i})-f_{t_o}({t_i})}\right)\right]\\
            &= \mathbb{E}_{t_i}[H(\pi_{t_i})] + \mathbb{E}_{{t_i}}\left[\log\left(\sum_je^{f_j({t_i})-f_{g({t_i})}({t_i})}\right)\right] + \mathbb{E}_{{t_i},{t_o}}[f_{g({t_i})}({t_i})-f_{t_o}({t_i})]\\
            &= \mathbb{E}_{t_i}[H(\pi_{t_i})] + \mathbb{E}_{{t_i}}\left[\log\left(\sum_je^{\lambda({t_i})(w_U(j)-w_U(g({t_i})))^Tw_U(g({t_i}))}\right)\right] + \mathbb{E}_{{t_i},{t_o}}[f_{g({t_i})}({t_i})-f_{t_o}({t_i})]
        \end{split}
    \end{equation*}
    For each ${t_i}$, $H(\pi_{t_i})\in [0,\log(M)]$ and \[\lambda \rightarrow \log\left(\sum_je^{\lambda(w_U(j)-w_U(g({t_i})))^Tw_U(g({t_i}))}\right)\] 
    is decreasing from $\log(M)$ to $0$. Thus there exists a solution $\lambda({t_i})$ to equation (1). Moreover since $w_u({t_o})w_U(j) \leq C$ for $j\neq {t_o}$ and $||w_U(g({t_i}))||^2_2\geq 1-C$,
    we obtain the bound \[\lambda({t_i}) \leq \frac{1}{1-2C}\log\left(\frac{M-1}{e^{H(\pi_{t_i})-1}}\right)\]
    Taking $\lambda({t_i})$ to be this solution 
    leaves us with
    \begin{equation*}
        \begin{split}
            d_{KL}(\pi, f) &= \mathbb{E}_{{t_i},{t_o}}[f_{g({t_i})}({t_i})-f_{t_o}({t_i})]\\
            &= \sum_{{t_i}}\pi({t_i})f_{g({t_i})}({t_i}) - \sum_{{t_i},{t_o}}\pi({t_i})\pi({t_o}|{t_i})f_{{t_o}}({t_i})\\
            &= \sum_{{t_i}}\pi({t_i})(1-\pi(g({t_i})|{t_i}))f_{g({t_i})}({t_i}) - \sum_{{t_i},{t_o}\neq g({t_i})}\pi({t_i})\pi({t_o}|{t_i})f_{{t_o}}({t_i})\\
            &= \sum_{{t_i}}\pi({t_i})(1-\pi(g({t_i})|{t_i}))\left(f_{g({t_i})}({t_i})+\frac{\sum_{{t_o}}f({t_i})_{t_o}}{M-1}\right) - \sum_{{t_i},{t_o}}\pi({t_i})\left(\pi({t_o}|{t_i})-\frac{(1-\pi(g({t_i})|{t_i}))}{M-1}\right)f_{{t_o}}({t_i})\\
            &\leq \sum_{{t_i}}\pi({t_i})(1-\pi(g({t_i})|{t_i}))\lambda({t_i})\left(1+2C+Cd_{TV}(\tilde{\pi}_{t_i}, \pi_{\text{unif}})\right)\\
            &\leq \mathbb{E}_{t_i}\left[(1-\pi(g({t_i})|{t_i}))\log\left(\frac{M-1}{e^{H(\pi_{t_i})}-1}\right)\left(\frac{1+2C+Cd_{TV}(\tilde{\pi}_{t_i}, \pi_{\text{unif}})}{1-2C}\right)\right]
        \end{split}
    \end{equation*}
    With $\tilde{\pi}_{t_i}({t_o}) = \pi({t_o}|{t_i},{t_o}\neq g({t_i}))$, and $\pi_{\text{unif}}$ the uniform distribution over $M-1$ tokens.
\end{proof}

\end{document}